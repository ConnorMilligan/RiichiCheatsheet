\documentclass[8pt,landscape]{article}
\usepackage{mahjong} % Display mahjong tiles
\usepackage{multicol} % Multiple columns
\usepackage{ifthen} % Conditional statements
\usepackage[landscape]{geometry} % Page geometry
\usepackage{CJKutf8} % Chinese, Japanese, Korean
\usepackage{array}
\usepackage{titlesec}
\titlespacing*{\section}{0pt}{0.5\baselineskip}{0.5\baselineskip}
\titlespacing*{\subsection}{0pt}{0.5\baselineskip}{0.5\baselineskip}

% Remove header and footer
\pagestyle{empty}
% Remove section numbering
\setcounter{secnumdepth}{0}

% Set page margins (https://github.com/wch/latexsheet/blob/gh-pages/latexsheet.tex)
\ifthenelse{\lengthtest { \paperwidth = 11in}}
	{ \geometry{top=.5in,left=.5in,right=.5in,bottom=.5in} }
	{\ifthenelse{ \lengthtest{ \paperwidth = 297mm}}
		{\geometry{top=1cm,left=1cm,right=1cm,bottom=1cm} }
		{\geometry{top=1cm,left=1cm,right=1cm,bottom=1cm} }
	}

\begin{document}
\begin{minipage}[t][.8\textheight][t]{\textwidth}
\begin{multicols*}{3}

    \begin{center}
        \begin{CJK}{UTF8}{min}
            \Large\textbf{リーチ麻雀 チートシート}
        \end{CJK}\\
        \small{Riichi Mahjong Cheat Sheet}
    \end{center}

    \section{The Game}
    Riichi Mahjong is a game for four players. 
    The goal is to get a winning hand by drawing and discarding tiles.
    The game ends when a player declares a winning hand.

    \subsection{The Hand}
    Similar to Omaha Poker, a player will swap out tiles from their hand to form a winning hand.
    The two main patters are sets and sequences.
    A set is three of the same tile, and a sequence is three consecutive tiles of the same suit.
    For example, \mahjong{123p} is a sequence, and \mahjong{111p} is a set.

    \section{The Tiles}
    There are 136 tiles in total; there are 34 kinds of unique tiles, with four of each kind.
    There are both numbered tiles and honor tiles.
    
    \subsection{Numbered Tiles}
    There are three suits of numbered tiles in a set of mahjong tiles, each with nine tiles numbered 1 to 9.
    The suits are: \textbf{Manzu} (\begin{CJK}{UTF8}{min}萬子\end{CJK}), 
    \textbf{Pinzu} (\begin{CJK}{UTF8}{min}筒子\end{CJK}), 
    and \textbf{Sōuzu} (\begin{CJK}{UTF8}{min}索子\end{CJK}).\\

    \subsection{Honor Tiles}
    There are two types of honor tiles:
    Three \textbf{Sangenpai} (\begin{CJK}{UTF8}{min}三元牌\end{CJK}) tiles \mahjong{675z} and four \textbf{Kazehai} (\begin{CJK}{UTF8}{min}風牌\end{CJK}) tiles \mahjong{1234z}.

    \section{Yaku}
    Yaku (\begin{CJK}{UTF8}{min}役\end{CJK}) are patterns that can be used to win the game.
    There are 46 Yaku in total, and each Yaku is worth a certain number of Han (\begin{CJK}{UTF8}{min}飜\end{CJK}).
    Only a couple basic Yaku are needed for most hands.

    \subsection{Riichi (Closed)}
    Riichi (\begin{CJK}{UTF8}{min}リーチ\end{CJK}) can be declared when a player is \textbf{one tile away from winning}.
    A player who has declared Riichi cannot change their hand.
    If a player wins after declaring Riichi, they will receive a bonus Han.

    \subsection{Tanyao}
    Tanyao (\begin{CJK}{UTF8}{min}断么九\end{CJK}) is a hand with no terminal or honor tiles.
    Terminal tiles are 1s and 9s, and honor tiles are \textbf{Sangenpai} and \textbf{Kazehai}.
    This is the most basic Yaku, and it is worth one Han.
    \mahjong{234m455667p23488s} is a Tanyao hand.

    \subsection{Pinfu (Closed)}
    Pinfu (\begin{CJK}{UTF8}{min}平和\end{CJK}) is a hand with no sets and a sequence as the winning tile.
    This sequence must be a two-sided wait (\begin{CJK}{UTF8}{min}両面待ち\end{CJK}).
    For example, \mahjong{12355m34789p456s} is a Pinfu hand. Because the winning tile is a double-sided wait (\mahjong{25p}).

    \section{Melding}
    Besides drawing and discarding tiles, a player can also meld tiles to form sets.
    These will take tiles from the player's hand and place them face up in front of the player.
    This will also open your hand, limiting certain Yaku and exposing your hand to other players.
    \subsection{Chi}
    Chi (\begin{CJK}{UTF8}{min}チー\end{CJK}) is a call to take a tile from the player to the right to complete a sequence.

    \subsection{Pon}
    Pon (\begin{CJK}{UTF8}{min}ポン\end{CJK}) is a call to take a tile from any player to complete a set.

    \subsection{Kan}
    Kan (\begin{CJK}{UTF8}{min}カン\end{CJK}) is a call to take a tile from any player and draw an extra tile.
    The dora indicator is flipped over, and the player who called Kan draws an extra tile.

    \subsection{Ron}
    Ron (\begin{CJK}{UTF8}{min}ロン\end{CJK}) is a call to win the game with a discard from any player.

    \section{Dora}
    Dora (\begin{CJK}{UTF8}{min}ドラ\end{CJK}) are bonus tiles that increase the value of a hand.
    The Dora indicator is the tile next to the dead wall.
    The Dora tile is the tile after the Dora indicator. For example, if the Dora indicator is \mahjong{5p}, then the Dora tile is \mahjong{6p}.
    Red Five tiles (\mahjong{0m0p0s}) are also Dora tiles.

    \section{Furi-Ten}
    Furi-Ten (\begin{CJK}{UTF8}{min}振聴\end{CJK}) is a penalty for discarding a tile earlier in the game that could have been used to win.
    If a player has Furi-Ten, they cannot win on a discard from another player.\\

    \rule{0.3\linewidth}{0.25pt}
    \scriptsize
    
    Created by Connor Milligan (2023).

    Licensed under GNU General Public License Version 3.
    

\end{multicols*}
\end{minipage}

% Bottom half of page
\begin{multicols*}{2}
    % Tile Table
    \hspace*{-1.5em}
    \begin{tabular}{m{0.2\linewidth}|c|c|c|c|c|c|c|c|c|}
    \multicolumn{1}{l|}{} & 1 & 2 & 3 & 4 & 5 & 6 & 7 & 8 & 9 \\ \hline
        \textbf{Manzu Tiles} (Characters) & 
            \raisebox{-0.4\height}{\resizebox{!}{2em}{\mahjong{1m}}} &
            \raisebox{-0.4\height}{\resizebox{!}{2em}{\mahjong{2m}}} &
            \raisebox{-0.4\height}{\resizebox{!}{2em}{\mahjong{3m}}} &
            \raisebox{-0.4\height}{\resizebox{!}{2em}{\mahjong{4m}}} &
            \raisebox{-0.4\height}{\resizebox{!}{2em}{\mahjong{5m}}} &
            \raisebox{-0.4\height}{\resizebox{!}{2em}{\mahjong{6m}}} &
            \raisebox{-0.4\height}{\resizebox{!}{2em}{\mahjong{7m}}} &
            \raisebox{-0.4\height}{\resizebox{!}{2em}{\mahjong{8m}}} &
            \raisebox{-0.4\height}{\resizebox{!}{2em}{\mahjong{9m}}} \\ \hline
        \textbf{Pinzu Tiles} (Circles) &
            \raisebox{-0.4\height}{\resizebox{!}{2em}{\mahjong{1p}}} &
            \raisebox{-0.4\height}{\resizebox{!}{2em}{\mahjong{2p}}} &
            \raisebox{-0.4\height}{\resizebox{!}{2em}{\mahjong{3p}}} &
            \raisebox{-0.4\height}{\resizebox{!}{2em}{\mahjong{4p}}} &
            \raisebox{-0.4\height}{\resizebox{!}{2em}{\mahjong{5p}}} &
            \raisebox{-0.4\height}{\resizebox{!}{2em}{\mahjong{6p}}} &
            \raisebox{-0.4\height}{\resizebox{!}{2em}{\mahjong{7p}}} &
            \raisebox{-0.4\height}{\resizebox{!}{2em}{\mahjong{8p}}} &
            \raisebox{-0.4\height}{\resizebox{!}{2em}{\mahjong{9p}}} \\ \hline
        \textbf{Sōuzu Tiles} (Bamboos) &
            \raisebox{-0.4\height}{\resizebox{!}{2em}{\mahjong{1s}}} &
            \raisebox{-0.4\height}{\resizebox{!}{2em}{\mahjong{2s}}} &
            \raisebox{-0.4\height}{\resizebox{!}{2em}{\mahjong{3s}}} &
            \raisebox{-0.4\height}{\resizebox{!}{2em}{\mahjong{4s}}} &
            \raisebox{-0.4\height}{\resizebox{!}{2em}{\mahjong{5s}}} &
            \raisebox{-0.4\height}{\resizebox{!}{2em}{\mahjong{6s}}} &
            \raisebox{-0.4\height}{\resizebox{!}{2em}{\mahjong{7s}}} &
            \raisebox{-0.4\height}{\resizebox{!}{2em}{\mahjong{8s}}} &
            \raisebox{-0.4\height}{\resizebox{!}{2em}{\mahjong{9s}}} \\ \hline
    \end{tabular}

    % Dragon Tiles
    \hspace*{-1.5em}
    \begin{tabular}{m{0.2\linewidth}|c|c|c|}
    \multicolumn{1}{c|}{} & Green & Red & White \\ \hline
        \textbf{Sangenpai} (Dragons) & 
            \raisebox{-0.4\height}{\resizebox{!}{2em}{\mahjong{6z}}} &
            \raisebox{-0.4\height}{\resizebox{!}{2em}{\mahjong{7z}}} &
            \raisebox{-0.4\height}{\resizebox{!}{2em}{\mahjong{5z}}} \\ \hline
    \end{tabular} \\ \\

    % Wind Tiles
    \hspace*{-1.5em}
    \begin{tabular}{m{0.2\linewidth}|c|c|c|c|}
    \multicolumn{1}{c|}{} & East & South & West & North \\ \hline
        \textbf{Kazehai} (Winds) & 
            \raisebox{-0.4\height}{\resizebox{!}{2em}{\mahjong{1z}}} &
            \raisebox{-0.4\height}{\resizebox{!}{2em}{\mahjong{2z}}} &
            \raisebox{-0.4\height}{\resizebox{!}{2em}{\mahjong{3z}}} &
            \raisebox{-0.4\height}{\resizebox{!}{2em}{\mahjong{4z}}} \\ \hline
    \end{tabular}
    
\end {multicols*}
\end{document}
